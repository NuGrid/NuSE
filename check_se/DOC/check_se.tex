% *********** Document name and reference:
% Title of document
\renewcommand{\ndoctitle}{Check\textunderscore se.py: Script for checking se output of MESA and NuGrid tools} 
% Document category acronym 
\renewcommand{\ndocname}{chem}                      
% svn dir
\renewcommand{\svndir}{svn://forum.astro.keele.ac.uk/utils/GCE/CHEM}  
% Contributors to this document
\renewcommand{\ndoccontribs}{CR,CF}

%\chapter{\ndoctitle}   % uncomment this if compiled into book in docs directory
\ngtitle{\ndoctitle}     % uncomment this is compiled into chapter wrapper this directory


Document name: \ndocname \\
SVN directory: \svndir\\
Contributors: \ndoccontribs\\



{  \textbf{Abstract:} \slshape
\ndocname\ allows to extensively check the se files for missing cycles and corrupted input.


%##############################################################
%# Section: Introduction
%##############################################################

\section{Introduction}
\index{\ndocname}
Welcome to the \ndocname.py user Guide. The purpose of \ndocname.py is to 
provide the user with a tool to do chemical evolution calculations.
The focus lies on simplification and user-friendliness. Therefore only basic
assumptions like instantaneous mixing are used.
Also there is no inflow or outflow.
However included are low- till massive star contributions, including
SN2 and SN1a.



This document will provide the user with a 
tutorial and walk through of the tools contained within \ndocname.py and
how a typical user would go about using and working with this python module.

\subsection{Environment}%Tutorial and Walkthough}

Start an interactive ipython session with
the mpython command defined as

\begin{verbatim}
alias mpython='ipython -pylab -p numpy'
\end{verbatim}
.

In order to set the ipython environment correctly,
make sure that the IPYTHON variable points to the
necessary python scripts.
The scripts chem.py and read\textunderscore yields.py
are needed and available on the NuGrid svn in:

\begin{verbatim}
../utils/GCE/CHEM.
\end{verbatim}

Also yield tables are available in CHEM as well as
other more choices in the yields\textunderscore tables
sub directory.


For bash:
\begin{verbatim}
export PYTHONPATH= ../utils/GCE/CHEM
\end{verbatim}
For tcsh:
\begin{verbatim}
setenv PYTHON ../utils/GCE/CHE
\end{verbatim}

When you do not use the NuGrid dir, make sure you have
all necessary scripts, e.g. nugridse.py and mesa.py within
your PYTHONPATH, as well as the modules, e.g. h5py, installed.



If the ipython executable is {\bf not} currently  located in your PATH variable a
\newline
\begin{verbatim}
 whereis ipython
\end{verbatim}
will provide you with the absolute location of the ipython executable.  


\subsection{Getting Started with GCE}

To following shows a short example.

Import the module:

\begin{verbatim}
>import sygma
\end{verbatim}

Next, initiate the class instance which is represented
by a executed GCE simulation:

\begin{verbatim}
>s=sygma.sygma(alphaimf=-2.35,dt=1e7,tend=1e10,mgal=1e9,
 table='yield_tables/isotope_yield_table.txt',
 sn1a_table='yield_tables/sn1a_t86.txt',
 bb_table='yield_tables/bb_walker91.txt')
\end{verbatim}

To analyze the run see the section about output.


\subsection{Input}

Input variables for the chem function are the following:

\begin{itemize}


  \item alphaimf: Setting the salpeter exponent for the initial mass function.
  \item iniZ: Define at which metallicity you want to start (possible Z= 0,0.02,0.01,0.006,0.001,0.0001)
  \item sfr: if 'input': Use sfr\textunderscore input file to read star formation for each timestep.
                         Specifies how much percent of ISM gas goes into stars. 
             if 'schmidt': Use the Schmidt birthrate B as specified in Timmes et al. 1995:
					$B = 2.8*m_{tot}* (m_{gas}/m_{tot})^2$ 
					where $m_{tot}$ and $m_{gas}$ is the total mass (density) 
					and the gas mass (density) respectively.

  \item dt: Time steps of the simulation (years)
	\item tend: Total time of simulation (years)
	\item mgal: Total mass content of simulation (Msun)
		
  \item table: NuGrid table with AGB and massive star yields (Msun)
  \item sn1a\textunderscore on: If true includes Sn1a, else not
  \item sn1a\textunderscore table: Table containing the SN1a yields (Msun)
  \item bb\textunderscore table: Table of initial abundances in mass fractions
		(e.g. Big Bang)
\end{itemize}


\subsection{Output}


To plot results, different functions are available.

To see the evolution of the mass fraction of elements or isotopes 
a plot() function with the following input parameter can be used:


\begin{itemize}
  \item xaxis/yaxis: Plot either 'age' for the time, 'Z' for the
	total metallicity, isotopes (format: 'H-1') or elements (format: 'H')
   \item source: Specifies if yields come from
            all sources ('all'), including
          AGB+SN1a, massive stars. Or from
          distinctive sources:
         only AGB stars ('agb'), only
          SN1a ('SN1a'), or only massive star ('massive')	
	\item norm : if 'no', no normalization of mass fraction,
                 if 'ini', normalization ot initial abundance

	
	
  \item fig: to name the plot figure
\end{itemize}

To plot the evolution of species in spectroscopic notation,
use the plot\textunderscore spec() function with the parameter:

\begin{itemize}
  \item xaxis: Use either 'age' for the time, or the
				spectroscopic notation (format: '[Fe/H]')
  \item yaxis: Use the spectroscopic notation (format: '[Fe/H]')
  
   \item source: Specifies if yields come from
            all sources ('all'), including
          AGB+SN1a, massive stars. Or from
          distinctive sources:
         only AGB stars ('agb'), only
          SN1a ('SN1a'), or only massive star ('massive')

  	\item label: Label for legend
  \item fig: to name the plot figure
\end{itemize}

The solar abundance is taken from GN93 (NuGrid file).

To get an overview over the available abundance distribution
(in mass fraction divided by solar mass fraction) of the ISM at a certain timestep a function
called plot\textunderscore abu\textunderscore distr() is available,
but still experimental.

\begin{itemize} 

\item t: time at which distribution is taken from, 
\item if -1: take abundance of last timestep (only -1 option is implemented)
\item x\textunderscore axis: if 'A' plot versus mass number A (only this implemented)		
\item solar\textunderscore norm: if True, use the solar normalization
\item marker1: choose marker type, 1,2,3
\item linest: choose linestyle, 1,2,3
\item y\textunderscore range: (optional) specify the y range
\item label: label of figure
\item fig: Figure name				
\end{itemize} 



To monitor the evolution of the ISM gas mass / ejected
masses over time use plot\textunderscore masses():

\begin{itemize} 

\item mass: either 'gas' for ISM/ejected mass or 'stars' for
		locked away mass
   \item source: Specifies if yields come from
            all sources ('all'), including
          AGB+SN1a, massive stars. Or from
          distinctive sources:
         only AGB stars ('agb'), only
          SN1a ('SN1a'), or only massive star ('massive')
\item norm: normalization either by using the total (initial gas)
                                mass mgal with 'ini', 
                                or the current total gas mass 'current'.
                                The latter case makes sense when comparing different
                                sources (see below)
\item log: if true use log y axis
\item fig: figure name
0,mass='gas',source='all',norm='ini',label='',log=False):

\end{itemize} 


To see the SFR over time the function
 


plot\textunderscore sfr()

plots the star formation rate
                        over time. If the input\textunderscore sfr file
                        is used plot this factor over time.
                        If a star formation formula is
                        used. e.g. 'schmidt' plots
                        the star formation rate [Msun/yr]


are available.



In the case you want to write out chemical evolution
of isotopes, use:

\begin{verbatim}
chem1.write_evol_table(table_name='gcetable.txt')
\end{verbatim}

\begin{itemize}
    \item table\textunderscore name:  specify name of the table
\end{itemize}

It writes out age, metallicity Z and isotopes
for each timestep (each timestep a line).

\subsection{Program structure}

The program details are explained in the following while
a flow chart of the running program is given in Fig. \ref{chart}

The main program is the chem class containing different
functions which are called when running/creating an instance
or afterwards for plotting or data extraction.

First the yield table files are read in using
classes in the file read\textunderscore yields.py.
The first one is a file with a 'Nugrid yield table' structure containing
yields of isotopes in solar masses plus lifetimes. Additionally,
a table containing SN1a yield data in solar masses is read.
If this file contains metallicity-dependence yields, this dependence
is adopted. Also a table file for initial abundances is read in (
Big Bang abundances) in mass fraction.

Then the program starts calculating the timesteps.
First the metallicity is of the ISM, consisting of initial abundances,
is calculated (in case of BB abundance it is 0, function getmetallicity).
Afterwards the star formation rate is calculated (function sfr).

Then the change of the ISM affected by the SFR is determined.
Important is hereby, that a part of the ISM is locked away in stars,
while (metal rich) stellar ejecta contributes to the ISM.
The ejecta is weighted by using an IMF and it is assumed
that all masses is ejected at the end of the lifetime of a star.


A delay time distribution (DTD) as specified
in Eq. 10 in Vogelsberger et al. (2013) is implemented in order
to calculate the number of SN1a events. The advantage of this
method is that progenitor, binary fraction and IMF information
is already contained. For each timestep with star formation
(new population), the contribution of SN1a ejecta of the
new population in all following timesteps is taken into account.


%For simplicity, SN1a occur for 1\% of AGB stars between 3Msun
%and 8Msun (Pritchet et al. 2008).
The function sfrmdot does all the steps mentioned above.

\subsection{Implementation details}

\subsubsection{The class history}

An extra class was introduced carrying
all the information about the history.
The idea is to clearly separate the 
simulation variables (needed to run
the code) and the variables saving
the results. Additionally, it is easier
to add further history variables since all
of them are at one location.

\subsubsection{ISM contribution sources}

In order to identify for each isotope and element
the contributing source(s) extra variables carry
the information about AGB star, SN1a and massive
star contribution. This makes the code clearly
slower, especially when calculating the elemental
contribution from the isotopic ones at the end
of the simulation.

\subsubsection{Dealing with the limited number of metallicities}

When given a certain ISM metallicity,the question arise which
models from the metallicity grid (and their yields) should be chosen.
Due to the lack of available simulations below Z=1e-4 (initial metallicity, set1.5a),
yields with initial metallicity of Z=1e-4 have to be used. In
the future Alex Heger's models might help to partly overcome this problem.
For super-solar cases the solar metallicity models are chosen.

For ISM Z values inside the grid, a interpolation of the grid yields
to the ISM metallicity has to be done.
There are different ways of interpolation. One simple way
is to linear interpolate between two metallicities of the grid 
$Z_i$, $Z_{i+1}$,
in order to get approximated yields $Y_{inter}$ at the ISM metallicity.

\begin{equation}
Y_{inter} = /(Y_i-Y_{i+1})/(Z_i-Z_{i+1}) * Z_{inter}
\end{equation}


Even though this method is used in Timmes et al. (1995)
it has clear disadvantages as explained in Wiersma et al (2009).
Another method, as motivated in the latter paper, is to split
the stellar ejecta into two components, the ones which are not
affected by the nucleosynthesis and the others which are 
produced or destroyed. As a result, a interpolation between
the ISM mass fraction and initial stellar mass fraction 
(simulation) for each isotope is done.

\begin{equation}
Eq. 6 in Wiersma et al (2009)
\end{equation}


The yield interpolation is done in the interpolate\textunderscore yields method,
in which the first, simpler method, is currently chosen.


The same assumption is used for the Z-dependent SN1a data.


\subsubsection{IMF and mass intervals}

With the factor $imf$ being the GCE input one can derive for a mass interval $i$
from $m_{a,i}$ to $m_{b,i}$ the number of stars $N_i$:


\begin{equation}
N_i = k \int_{m_a}^{m_b} m^{-2.35} dm = p ( m_b^{-1.35} - m_a^{-1.35} )
\end{equation}

with $k$ being the normalization constant and $p = -k/1.35$.
The total mass in the interval can be calculated by using 

\begin{equation}
M_i = k \int_{m_a}^{m_b} m^{-2.35}\, m\, dm = p^* ( m_b^{-0.35} - m_a^{-0.35} )
\label{eq:imf}
\end{equation}

with $p^* = -k/0.35$.

Due to the fact that only a limited number of simulated masses
need to covering the whole (chosen) mass range of the IMF it is necessary
to assign each mass a certain mass interval of the IMF.
The number of stars occupying each interval are then represented
by the available simulated mass $m_i$ in the interval and its yields.

Then it follows for the mass $M_i$ of a interval i
with available (simulated) star mass $m_i$:
 
\begin{equation} 
M_i=p ( m_{b,i}^{-1.35} - m_{a,i}^{-1.35} ) * m_i = N_i * m_i
\end{equation}

Being $m_{l}$ and $m_{u}$ now the outer boundaries of the total mass range
under consideration,
covered by a number of mass intervals i with available star mass $m_i$,
then the total mass is given by:

\begin{equation}
M_{tot} = p \sum\limits_{i}( m_{b,i}^{-1.35} - m_{a,i}^{-1.35} ) * m_i
\end{equation}


In order to extend the total mass range, chosen as $m_{l}=0.5$ and $m_{u}=40$,
to $m_{l*}=0.1$ and $m_{u*}=100$ (to the more quoted limits) one has to use the factor $fac$,
which makes use of Eq. \ref{eq:imf}:


\begin{equation}
fac = \frac{M_{tot}}{M^*_{tot}}
= \frac{k \int_{m_l}^{m_u} m^{-2.35}\, m\, dm }{k \int_{m_{l*}}^{m_{u*}} m^{-2.35}\, m\, dm}
= \frac{ ( m_u^{-0.35} - m_l^{-0.35} ) } { ( m_{l*}^{-0.35} - m_{u*}^{-0.35} ) }
\end{equation}

The weight (number of stars) $massfac$, given a specific
system (gas cloud) of mass $M_c$, for each simulated mass $m_i$ is then:


\begin{equation}
massfac_i = M_c * fac * \frac{N_i}{ M_{tot} }
\end{equation}

The yield contribution $Y_{i,iso}$ of a simulated star mass $i$ 
of a certain isotope $iso$ can then be used to calculate
the total mass of the cloud $Y_{c,i,iso}$:

\begin{equation}
Y_{c,i,iso} = Y_{i,iso} * massfac_i
\end{equation}

\subsubsection{Comment on SFR and input methods}

Due to the fact that this is a 1-D 'box'-model,
but star formation rates in literature are 
typically derived and given in mass per time
and per surface or volume density, the current
approach might be questionable.

In this code, the dependency of the SFR from
the area is ignored and the literature values
taken as mass per time. 


\subsection{Disclaimer}

		
\subsection{History} 
This document history complements the svn log.

\begin{tabular*}{\textwidth}{lll}
\hline
Authors & yymmdd & Comment \\
\hline
CR & 130926 & creation \\
\end{tabular*}


\subsection{Contact}
If any bugs do appear or if there are any questions, please email critter@uvic.ca
% --------------- latex template below ---------------------------
\begin{verbatim}

\end{verbatim}


%\begin{figure}[htbp]
%   \centering
%%   \includegraphics[width=\textwidth]{layers.jpg} % 
%      \caption{}   \includegraphics[width=0.48\textwidth]{FIGURES/HRD90ms.png}  
%   \includegraphics[width=0.48\textwidth]{FIGURES/HRD150ms.png}  
%
%   \label{fig:one}
%\end{figure}
%
%\begin{equation}
%Y\_a = Y\_k + \sum\_{i \neq k} Y\_i
%\end{equation}
%
%{
%%\color{ForestGreen}
%\sffamily 
%  {\center  --------------- \hfill {\bf START: Some special text} \hfill ---------------}\\
%$Y\_c$ does not contain ZZZ but we may assign one $Y_n$ to XYZ which is the decay product of the unstable nitrogen isotope JJHJ. %
%
%{\center ---------------  \hfill {\bf END:Some special text} \hfill ---------------}\\
%}
